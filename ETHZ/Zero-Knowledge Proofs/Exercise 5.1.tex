\documentclass{scrartcl}
\usepackage[margin=1in,a4paper]{geometry}
\usepackage[utf8]{inputenc}
\usepackage{amsfonts}         % for natural N
\usepackage{amsmath}          % for \text{}
\usepackage{indentfirst}      % to indent first line of section
\usepackage{hyperref}         % for \nameref{}
\usepackage{titlesec}         % for text next to section
\usepackage{mathabx}

\renewcommand{\thesubsection}{}
\setlength{\parskip}{0.5em}   % spacing between paragraphs
\setlength\parindent{0pt}     % no indent at start of paragraphs

\titleformat{\section}[hang]
  {\normalfont\Large\bfseries}{\thesection}{1em}{}
\titleformat{\subsection}[runin]
  {\normalfont\large\bfseries}{\thesubsection}{1em}{}

\title{Exercise 5}
\subtitle{Zero-Knowledge Proofs}
\author{Loïc Baccigalupi}

\begin{document}

\maketitle

\section*{5.1 Public-Coin Interactive Proof for Graph Non-Isomorphism}

\subsection*{a)}

Because if $v \in \mathbb{F}_{2^m}$ is uniformly random, then we can see that each element $v_i$ ($v = (v_1, \dots, v_m)$) is also uniformly random. 
This implies that the vector $[v] = (v_1, \dots, v_k)$ is also uniformly random, as it is composed of uniformly random vector elements.

Because of this, we know that $y$ and $y^{\prime}$ are independent.

This in turn shows that $\forall x \neq x^{\prime} \in \mathbb{F}_{2^m}$ and $y,y^{\prime} \in \mathbb{F}_{2^k}$:
\begin{equation*}
    \begin{split}
        \text{P}_{h \leftarrow \mathcal{H}}\{h(x)=y \wedge h(x^{\prime}) = y^{\prime} \}
            &= \text{P}_{h_{a,b} \leftarrow \mathcal{H}}\{[ax+b]=y \wedge [ax^{\prime} + b] = y^{\prime} \} \\
            &= \text{P}_{h_{a,b} \leftarrow \mathcal{H}}\{[ax+b]=y \}\text{P}_{h_{a,b} \leftarrow \mathcal{H}}\{[ax^{\prime} + b] = y^{\prime} \} \\
            &= \frac{1}{2^k}\frac{1}{2^k} \\
            &= 2^{-2k}
    \end{split}
\end{equation*}


\subsection*{b)}

Let $P = \text{P}_{h, y}\{\exists x \in S : h(x) = y \} $ and $T = \{h(x) : x \in S \} \subseteq \mathbb{F}_{2^k}$ the set of images of $h$ where the inputs are elements of $S \subseteq \mathbb{F}_{2^m}$ .

The cardinality $|T|$ is at a maximum if every element of $S$ maps to a distinct element in $\mathbb{F}_{2^k}$, i.e. $|T|=|S|$.
This implies that the upper bound of $P$ is the probability of picking a random element of $T$, i.e.:

\begin{equation*}
    P \leq \frac{|T|}{2^k} = \frac{|S|}{2^k} = |S|2^{-k}
\end{equation*}

Because we used the maximum cardinality $|T|$, this implies we used the upper bound of $p\ (= N2^{-k})$. By knowing that $|[\frac{1}{4}; \frac{1}{2})| = \frac{1}{4}$, we can find the lower bound of $P$:

\begin{equation*}
    P \leq |S|2^{-k}\left(1- \frac{1}{4} \right) = 3|S|2^{-k-2}
\end{equation*}


\subsection*{c)}

The prover can find with probability at least $3|S|2^{-k-2} = \frac{3}{4}p$ when $|S| = N$ the correct $s$ to send. This means that the protocol is $\frac{3}{4}p$-complete.

If $|S| \leq \frac{N}{2}$, the probability of finding the correct $x$ is at most half of what it is when $|S| = N$ ($|S|$ is at most half of $N$). So the protocol is $\frac{p}{2}$-sound. 


\subsection*{d)}

We can use the GI zero-knowledge proof as seen in class to prove that a graph $X$ is isomorphic to $G_0$ or $G_1$, i.e. $X \in S$.

If $G_0 \cong G_1$, $|S| = n!$ and if $G_0 \ncong G_1$, $|S| = 2n!$.

\subsection*{e)}

If we set $N = 2n!$ and $S$ as seen in the previous sub-question, we want our protocol to accept if $|S| = 2n! = N$ and reject if $|S| \leq n! = \frac{N}{2}$ (same conditions as in Goldwasser-Sipser).

Now we have that $S = \{H : H \cong G_0 \ \text{or}\ H \cong G_1 \} \subseteq \{\text{"All graphs with}\ n \ \text{vertices"}\}$

\end{document}
